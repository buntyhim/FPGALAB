\documentclass[10pt,a4paper]{article}
\usepackage[utf8]{inputenc}
\usepackage{amsmath}
\usepackage{amsfonts}
\usepackage{multicol}
\usepackage{amssymb}
\usepackage[framed]{matlab-prettifier}
\usepackage{graphicx}
\usepackage[margin=0.75in]{geometry}
\usepackage{enumerate}
\usepackage{circuitikz}

\begin{document}
\begin{center}

{\huge EE5811 : FPGA LAB}\\
{\large ASSIGNMENT 1}

\end{center}
Himanshu Yadav \hfill \today \\ IS21MTECH11003

\vspace{15pt}
\hrule
\vspace{5pt}


\section*{Problem}

Question 6) b) from papers/cbse/cs/2017.pdf\\
Draw the logic circuit of the following boolean expression using only NOR Gates :

\begin{center}
    ( A + B ) . ( C + D )
\end{center}


\vspace{15pt}
\hrule
\vspace{5pt}

\section*{Solution}
To convert the above expression to NOR only logic circuit we will take the Double conjugate of the whole expression and then use de Morgan's Theorem to simplify it.

\begin{center}
   $( A + B ) . ( C + D )$\\
   \vspace{5pt}
   $\overline{\overline{( A + B ) . ( C + D )}}$
   \\
   \vspace{5pt}
   $\overline{\overline{( A + B )} + \overline{( C + D )}}$
    
\end{center}
The last expression can easily be drawn using NOR gates in the following manner

\begin{center}
    
\begin{circuitikz}
\draw
(0,0)node[nor port](mynor1){}

(0,2)node[nor port](mynor2){}

(3,1)node[nor port](mynor3){}

(mynor1.out)|-(mynor3.in 2)
(mynor2.out)|-(mynor3.in 1);

\node at (-1.6,0.3) {$C$};
\node at (-1.6,1.7) {$B$}; 
\node at (-1.6,-0.3) {$D$};
\node at (-1.6,2.3) {$A$};
\node at (1,2) {$\overline{( A + B )}$};
\node at (1,0) {$\overline{( C + D )}$};
\node at (5,1) {$\overline{\overline{( A + B )} + \overline{( C + D )}}$};

\end{circuitikz}

\end{center}
A C Code has been attached in the repository for verification \label{code:nor_representation}
\begin{lstlisting}
./nor_representation.c
\end{lstlisting}


\end{document}